\documentclass{article}
\usepackage[utf8]{inputenc}
\usepackage{amsmath,amsfonts,amssymb,amsthm}
%\usepackage{mathtools}
\usepackage{graphicx}
\usepackage{setspace}
% \usepackage[a4paper, total={6in, 8in}]{geometry}
\usepackage[top=2.5cm, left=2.0cm, right=2.0cm, bottom=3.0cm]{geometry}

\title{%
     Business Cycles: Empirics and Theory
}
\author{Wenjie Tu}
\date{Fall Semester 2021}

\setlength{\parindent}{0pt}
%\setlength{\parskip}{1em}
%\onehalfspacing
\begin{document}

\maketitle

\section{Introduction}

Some jargons:

\begin{itemize}
    \item ar = annual rate
    \item na = not adjusted
    \item sa = seasonally adjusted (Swiss case: sa = sport-event adjusted)
    \item q/q sa quarterly seasonally adjusted growth
    \item q/q csa quarterly seasonally and calendar adjusted growth
    \item q/q saar = quarterly seasonally growth expressed at an annual rate
    \[\textrm{q}/\textrm{q saar} = (1+\textrm{q/q sa})^4 - 1\]
    \item y/y = growth relative to the same quarter of the previous year (usually calculated using na series)
\end{itemize}

Some examples:

\begin{itemize}
    \item $Y_\text{Q2 2018}$ = level of real GDP in Q2 2018
    \item $4\times Y_\text{Q2 2018}$ = level of real GDP in Q2 2018, expressed at an annual rate
    \item $Y_\text{Q2 2018}^{sa}$ = seasonally adjusted level of real GDP in Q2 2018
    \item q/q growth in Q2 2018 (sa) = $\left(\dfrac{Y_\text{Q2 2018}^{sa}}{Y_\text{Q1 2018}^{sa}}\right)^4-1$
    \item y/y growth = $\dfrac{Y_\text{Q2 2018}}{Y_\text{Q2 2017}}-1$
    \item annual growth = $\dfrac{Y_\text{Q1 2018}+Y_\text{Q2 2018}+Y_\text{Q3 2018}+Y_\text{Q4 2018}}{Y_\text{Q1 2017}+Y_\text{Q2 2017}+Y_\text{Q3 2017}+Y_\text{Q4 2017}}-1$
\end{itemize}

Some notations:

\begin{itemize}
    \item $X_t$ = level in period $t$, e.g., GDP during Q1 2018
    \item $x_t=\log(X_t)$ = natural logarithm of $X_t$
    \item $\Delta x_t=x_t-x_{t-1}$
    \item The growth rate of $X_t$
    \[\dfrac{X_t-X_{t-1}}{X_{t-1}}\simeq\log\dfrac{X_t}{X_{t-1}}=\Delta x_t \]
\end{itemize}

Some useful approximations:

\begin{itemize}
    \item y/y growth = 4 quarter moving average of q/q ar growth = sum of q/q growth rates = $\Delta x_t+\Delta x_{t-1}+\Delta x_{t-2}+\Delta x_{t-3}$
    \item Some definitions to write down approximations to annual growth rates
    \begin{itemize}
        \item $g_{t,q}^{q/q}\equiv$ q/q growth rate in $q$th quarter of the year $t$
        \item $g_{t,q}^{y/y}\equiv$ y/y growth rate in $q$th quarter of the year $t$
        \item $g_{t,q}^{A}\equiv$ annual growth rate in year $t$
    \end{itemize}
    \item The following approximations hold
    \begin{itemize}
        \item $g_t^A\simeq\frac{1}{4}(g_{t,4}^{q/q}+2g_{t,3}^{q/q}+3g_{t,2}^{q/q}+4g_{t,1}^{q/q}+3g_{t-1,4}^{q/q}+2g_{t-1,3}^{q/q}+g_{t-1,2}^{q/q})$
        \item $g_t^A\simeq\frac{1}{4}(g_{t,4}^{y/y}+g_{t,3}^{y/y}+g_{t,2}^{y/y}+g_{t,1}^{y/y})$
    \end{itemize}
\end{itemize}

Important implications of approximations to annual growth rate

\begin{itemize}
    \item Annual growth is not the average of the quarterly growth rates of the respective year.
    \item Annual growth depends also on quarterly growth in the previous year.
    \item In fact, the previous year's Q4 growth rate is three times as important for this year's annual growth than this year's Q4 growth rate.
    \item Once you know growth up to Q2, you can already make a very precise forecast for annual growth (you already know 13/16 of the annual growth rate).
    \item Similar effect for y/y growth rates (you always know already 3/4 of the next y/y growth rate).
\end{itemize}

\section{Detrending}

\textbf{Linear trend}

\[\log(Y_t)=\beta_0+\beta_1 t+\varepsilon_t \]

\begin{itemize}
    \item $\beta_1$ is the trend growth rate.
    \item $\varepsilon_t$ are the deviations from the trend and therewith the cycle.
\end{itemize}

\textbf{Time-varying trend}

\[c_t\equiv\frac{Y_t-Z_t}{Z_t} \]

\begin{itemize}
    \item $Z_t$: the ``trend level'' of GDP.
    \item $c_t$: the percentage deviation of GDP from trend by $c_t$.
\end{itemize}

Using log-approximations:

\[c_t=\frac{Y_t}{Z_t}-1\simeq\log\frac{Y_t}{Z_t}=y_t-z_t \]

\[y_t=\log(Y_t)=z_t+c_t \]

i.e., log-level of GDP consists of a trend $z_t$ and a cyclical component $c_t$ that fluctuates around the trend.

\vspace{4mm}

\textbf{How to govern the degree of time variation in trend growth}

Trend growth rate:

\[\frac{Z_t}{Z_{t-1}}-1\simeq\log\frac{Z_t}{Z_{t-1}}=z_t-z_{t-1} \]

The change of the trend growth rate:

\[v_t=(z_t-z_{t-1})-(z_{t-1}-z_{t-2}) \]

If we allow for a lot of time-variation in the trend growth rate, we can choose $v_t$ such that most of the observed changes in $\log(Y_t)$ is attributable to changes in $z_t$ and only little to cyclical component $c_t$ and vice versa.

\vspace{4mm}

\textbf{HP filter}

\begin{align*}
    L&=\sum_t c_t^2 + \lambda\sum_t v_t^2 \\
    &=\sum_t (\log(Y_t)-z_t)^2 + \lambda\sum_t ((z_t-z_{t-1}) - (z_{t-1}-z_{t-2}))^2
\end{align*}

\[\hat{z}_t=\underset{z_t}{\arg\min}\quad L \]

For $\lambda\to0$:

\[\min_{z_t} L=\min_{z_t}\sum_t(\log(Y_t)-z_t)^2 \]

\begin{itemize}
    \item The time trend $z_t$ that minimizes the loss function is simply $\log(Y_t)=z_t$.
    \item There is no cycle!
\end{itemize}

For $\lambda\to\infty$:

\[\min_{z_t} L=\min_{z_t}\sum_t ((z_t-z_{t-1}) - (z_{t-1}-z_{t-2}))^2 \]

\begin{itemize}
    \item The loss function is minimized with $z_t-z_{t-1}=z_{t+j}-z_{t+j-1}\quad\forall t, j$.
    \item Trend growth is constant, i.e., linear time trend.
\end{itemize}

Problems with HP filters:

\begin{itemize}
    \item Instability of the HP trend at the margin.
    \item HP filter cannot separate demand from supply shocks (i.e., a slowdown in trend growth is identified during a long-lasting recession).
    \item What is the right $\lambda$ for HP filter.
\end{itemize}

\textbf{Production function approach}

Cobb-Douglas production function:

\[Y=AK^{\alpha}L^{1-\alpha} \]

\begin{itemize}
    \item Measure $K$ and $L$ (which is not trivial) and solve for $A$.
    \item Filter $A, K, L$ to get their potential levels.
    \item Plug these back into production function to get potential GDP.
\end{itemize}

\textbf{Another way of writing the HP filter}

Assume that the change in the trend is a random variable:

\[\Delta z_t-\Delta z_{t-1}=v_t, v_t\sim\mathcal{N}(0,\sigma^2) \]

\[(z_t-z_{t-1})-(z_{t-1}-z_{t-2})=v_t \]


\[
\underbrace{
\begin{bmatrix}
    z_t \\ z_{t-1}
\end{bmatrix}
}_{Z_t}
=
\underbrace{
\begin{bmatrix}
    2 & -1 \\
    1 & 0
\end{bmatrix}
}_{F}
\underbrace{
\begin{bmatrix}
    z_{t-1} \\ z_{t-2}
\end{bmatrix}
}_{Z_{t-1}}
+
\underbrace{
\begin{bmatrix}
    \sigma & 0 \\
    0 & 0
\end{bmatrix}
}_{Q}
\underbrace{
\begin{bmatrix}
    \varepsilon_t^{\Delta z} \\ 0
\end{bmatrix}
}_{w_t}
\]

\[Z_t=FZ_{t-1}+Qw_t, w_t\sim\mathcal{N}(0, 1) \]

We do not observe the trend $Z_t$, but we observe data that is a function of the unobserved trend:

\[
\log(Y_t)=z_t+c_t=
\underbrace{
\begin{bmatrix}
    1 & 0
\end{bmatrix}
}_{H}
\underbrace{
\begin{bmatrix}
    z_t \\ z_{t-1}
\end{bmatrix}
}_{Z_t}
+
\underbrace{
\begin{bmatrix}
    \sigma & 0
\end{bmatrix}
}_{R}
\underbrace{
\begin{bmatrix}
    \lambda\varepsilon_t^c \\ 0
\end{bmatrix}
}_{e_t}
\]

\[
\log(Y_t)=HZ_t+Re_t, e_t\sim\mathcal{N}(0, 1)
\]

Note: variance of $c_t$ is $\lambda$ times larger than variance of $v_t$.

\vspace{4mm}

\textbf{The state-space form}

\underline{Measurement equation} with standard normal i.i.d. shocks $e_t$:

\[\log(Y_t)=X_t=HZ_t+Re_t \]

\underline{Transition equation} with standard normal i.i.d. shocks $w_t$:

\[Z_t=FZ_{t-1}+Qw_t \]

Note: 

\begin{itemize}
    \item Whenever, we can write something in state-space form, we can use \textit{Kalman filter} to estimate the unobserved states $Z_t$ given the parameter matrices $H, R, F, Q$ and the data $X_t$.
    \item \textit{Kalman filter} can easily handle missing data.
\end{itemize}

\section{Recessions v.s. Expansions}

\textbf{Recessions}

\begin{itemize}
    \item Periods with contracting economic activity.
    \item Popular rule of thumb: two consecutive quarters of negative real GDP growth.
\end{itemize}

\textbf{Regime switching model (Hamilton filter)}

\begin{itemize}
    \item Assume that there are two regimes $S_t=1, 2$
    \item Assume that the economy stochastically switches between the two regimes with the following probabilities
    \begin{center}
        \includegraphics[scale=0.2]{state-space.png}
    \end{center}
\end{itemize}

Regimes differ in average growth:

\[
\Delta y_t = 
\begin{cases}
c_1+\varepsilon_t & S_t=1 \\
c_2+\varepsilon_t & S_t=2
\end{cases},
\varepsilon_t\sim\mathcal{N}(0,\sigma^2)
\]

\textbf{Comparison}

Linear view:

\begin{itemize}
    \item Business cycles are fluctuations around a trend.
    \item Most modern macro models (DSGEs) implicitly have this view.
\end{itemize}

Non-linear view:

\begin{itemize}
    \item Business cycles are transitions between two (or more) distinct regimes.
    \item While there are fluctuations within a regime, what really matters is the transition between the regimes.
    \item The public and high-level policy makers often take this view.
\end{itemize}

\textbf{One of the best forecasting indicators for recessions: the yield curve}

\begin{itemize}
    \item Yield curve slope is historically one of the best indicators for upcoming recessions.
    \item All recessions were preceded by an inverted (negatively sloped) yield curves, yield curve was rarely inverted without a recession following.
\end{itemize}

\textbf{Why is an inverted yield curve a good recession predictor?}

Assume you have 100 USD to invest for two years

There are two options:

\begin{itemize}
    \item Option 1: buy a 2-year treasury bond with a yield of $i_t^{2y}$, get the principal payment $2(100(1+i_t^{2y})^2)$ after 2 years.
    \item Option 2: buy a 1-year treasury bond with a yield of $i_t^{1y}$, get the principal payment $100(1+i_t^{1y})$ after 1 year and reinvest it in a new 1-year treasury bond with an expected yield $\mathbf{E}\left[(1+i_{t+1}^{1y})\right]$. You expect to have $100(1+i_t^{1y})\mathbf{E}\left[(1+i_{t+1}^{1y})\right]$ after 2 years.
\end{itemize}

Except for the uncertainty about $i_{t+1}^{1y}$, the two options are very similar, therefore the return has to be similar

\[
100(1+i_t^{2y})^2\approx 100(1+i_t^{1y})\mathbf{E}\left[(1+i_{t+1}^{1y})\right]
\]

Take logs of both sides:

\[
2\log(1+i_t^{2y})\approx\log(1+i_t^{1y})+\log\left(1+\mathbf{E}\left[i_{t+1}^{1y}\right] \right)
\]

Use the approximation equation $\log(1+x)=x$:

\[
i_t^{2y}\approx\frac{1}{2}\left(i_t^{1y}+\mathbf{E}\left[i_{t+1}^{1y}\right] \right)
\]

Similarly, for 10-year yield:

\[
i_t^{10y}\approx\frac{1}{10}\sum_{i=0}^{9}\mathbf{E}_t \left[i_{t+i}^{1y}\right]
\]

\begin{itemize}
    \item Short-term yields are mostly driven by monetary policy ($i_t^{1y}\approx\text{policy rate}$).
    \item An inverted yield curve implies that the policy rate is currently higher than what is expected in the future.
    \item In other words, market participants expect the central bank to lower the policy rate.
    \item Central banks typically only lower policy rates in response to economic weakness.
    \item Word of caution: the risk-premium may lead to an inverted yield curve, without market participants expecting falling policy rates
    \[i_t^{10y}=\frac{1}{10}\sum_{i=0}^{9}\mathbf{E}_t i_{t+i}^{1y}+\text{risk premium}_t \]
\end{itemize}

Recession nowcasting is feasible!

Good rule of thumb: a recession has started if 3-month change of the 3-month moving average of seasonally adjusted unemployment rate rises above 0.3 pp.

\textbf{Important references: baseline recession probabilities}

NBER recession

\begin{itemize}
    \item Time span: from Q1 1950 until Q2 2021 (286 quarters in total)
    \item \# of expansion quarters = 247
    \item \# of recession quarters = 39
    \item Unconditional recession probability = $\dfrac{39}{286}=14\%$
    \item \# of expansion quarters that are followed by a recession = 11
    \item Conditional of being in an expansion quarter:
    \begin{itemize}
        \item probability of a recession starting next quarter = $\dfrac{11}{247}=4.5\%$
        \item probability of staying in an expansion = $100-4.5=95.5\%$
    \end{itemize}
    \item Conditional on being in an expansion in $t$:
    \begin{itemize}
        \item probability of next recession starting sometimes during next 2 quarters ($t+1$ and $t+2$) = $1-0.955\times0.955$
        \item probability of next recession starting sometimes during next $n$ quarters (between $t+1$ and $t+n$) = $1-0.955^n$
    \end{itemize}
\end{itemize}

\section{Welfare Costs of Business Cycles}


HP decomposition: $\log C_t=z_t+x_t$ (or equally: $C_t=Z_tX_t$), where $z_t$ is the trend level of consumption and $x_t$ fluctuations around this trend.

The Lucas calculation - approximating the utility function

\[U(C)\simeq U(Z)+U'(Z)(ZX-Z)+\frac{1}{2}U''(Z)(ZX-Z)^2 \]

Calculating expected utility:

\[\mathbf{E}[U(C)]\simeq U(Z)+U'(Z)\mathbf{E}[(ZX-Z)]+\frac{1}{2}U''(Z)\mathbf{E}\left[(ZX-Z)^2 \right] \]

Rewriting:

\[\mathbf{E}[U(C)]\simeq U(Z)+\frac{1}{2}U''(Z)\mathbf{E}\left[(X-1)^2 \right]Z^2 \]

Utility function:

\[U(C)=\frac{C_{1-\sigma}}{1-\sigma} \]

Hence,

\[\mathbf{E}[U(C)]\simeq\frac{Z^{1-\sigma}}{1-\sigma}-\frac{\sigma}{2}Z^{-\sigma-1}\mathbf{E}\left[(\underbrace{X-1}_{\simeq\log X=x})^2 \right]Z^2 \]

\[\mathbf{E}[U(C)]\simeq\frac{Z_{1-\sigma}}{1-\sigma}-\frac{\sigma}{2}Z^{1-\sigma}\mathbf{E}[x^2] \]

\begin{itemize}
    \item $\mathbf{E}[x^2]$ is the variance of log-difference ($\simeq$ percentage deviation) of consumption from its trend level
    \item The higher the variance, the lower expected utility
    \item The higher risk aversion, $sigma$, the more costly are fluctuations
\end{itemize}

Certainty equivalence

Let us scale trend level of consumption by $\delta$ and offer $\delta Z$ with certainty to the representative consumer

\[\frac{(\delta Z)^{1-\sigma}}{1-\sigma}=\frac{Z_{1-\sigma}}{1-\sigma}-\frac{\sigma}{2}Z^{1-\sigma}\mathbf{E}[x^2] \]

Rearrange:

\[\delta=\left(1-(1-\sigma)\frac{\sigma}{2}\mathbf{E}[x^2] \right)^{\frac{1}{1-\sigma}} \]

The resulting estimate for the costs of economic fluctuations is tiny, but why should we care about business cycles?

\begin{itemize}
    \item Observed fluctuations are the ones resulting despite macroeconomic stabilization policies (might be much larger without stabilization)
    \item Cost of business cycles are not equally distributed over population (some become unemployed, some not; not everyone becomes 4\% unemployed, i.e. can work 4\% fewer hours)
    \item Direct costs of unemployment
    \item Persistent effects of economic fluctuations
\end{itemize}

\section{Comovement}

\textbf{Principal Component Analysis}

\[y_{it}=\lambda_i f_t+\varepsilon_{it} \]

\begin{itemize}
    \item $y_{it}$: the standardized value of the indicator $i$ at time $t$
    \item $f_t$: the underlying force that drives the common variation in the dataset
    \item $\lambda_i$: the ``loading'' of indicator $i$ on $f_t$
\end{itemize}

Assume that we know the loadings $\lambda_i$:

\[y_t=
\begin{bmatrix}
    y_{1t} \\ \vdots \\ y_{Nt}
\end{bmatrix}=
\begin{bmatrix}
    \lambda_1 \\ \vdots \\ \lambda_N
\end{bmatrix}f_t + 
\begin{bmatrix}
    \varepsilon_{1t} \\ \vdots \\ \varepsilon_{Nt}
\end{bmatrix}
\]

For given loadings $\lambda$, we can obtain $f_t$ by simply running a regression in each time period $t$

\[\hat{f}_t=(\lambda'\lambda)^{-1}\lambda'y_t \]

\[\hat{\varepsilon}_t=y_t-\lambda(\lambda'\lambda)^{-1}\lambda'y_t \]

The sum of squared errors:

\[\sum_t\hat{\varepsilon}_t'\hat{\varepsilon}_t=
\sum_t\left(y_t-\lambda(\lambda'\lambda)^{-1}\lambda'y_t\right)'\left(y_t-\lambda(\lambda'\lambda)^{-1}\lambda'y_t\right)
\]

\section{Swiss GDP}

\textbf{How to estimate Swiss GDP}
\begin{itemize}
    \item Annual GDP adds up the value added according to firm survey
    \item Value added is estimated from the production side
    \item Demand side: inventories as residual
    \item Income side: profits as residual
    \item First estimate published 8 months after the end of reference year
    \item i.e. first estimate for 2021 will be published at the end of August 2022
\end{itemize}

\textbf{Quarterly GDP is an intra- and extra-polation of annual GDP}
\begin{itemize}
    \item Regress annual value added of a sector $y_t^A$ on indicators
    \[y_t^A=\beta_0+\beta_1 x_t^A+\varepsilon_t \]
\end{itemize}

\section{Introduction to Theory}

\textbf{IS curve}: goods \& services market equilibrium

Aggregate demand equals investment, private consumption and government spending (closed economy):

\[Y^D = I + C + G \]

Aggregate demand depends on aggregate income $Y^I$, interest rate, $r$, and other factors, $v_D$

\[Y^D = f(\underset{+}{Y^I}, \underset{-}{r}) + v_D \]

In equilibrium: 

\[\underbrace{\textrm{aggregate demand}}_{Y^D}=\underbrace{\textrm{aggregate income}}_{Y^I}=\underbrace{\textrm{aggregate production}}_{Y^P} \]

\textbf{LM curve}: money market

Monetary policy rule (MP-rule)

\[r=\rho+\theta_{\pi}\pi+v_{MP} \]

\begin{itemize}
    \item $\pi$: inflation
    \item $v_{MP}$: monetary policy shock
\end{itemize}

The IS-LM model provides the equilibrium combination of interest rate and inflation for a given demand and monetary policy shock.

\begin{itemize}
    \item The IS-LM model describes goods and money market equilibrium in the short-term
    \item Inflation is assumed to be constant
\end{itemize}

\textbf{AS-AD model}:

\begin{itemize}
    \item The AS-AD model is an extension of the IS-LM framework
    \item It describes the relationship between aggregate output and prices
\end{itemize}

\textbf{AD curve}: aggregate demand

\begin{itemize}
    \item The AD curve is defined by the IS-LM equilibrium
    \item For given inflation rate, $\pi$, the MP rule describes, what interest rate, $r$, should be chosen by the central bank
    \item From IS curve, we can get the resulting aggregate demand, $Y$
    \item The AD curve describes the resulting aggregate demand for each inflation rate
    \item The monetary policy shock and the demand shock shift the AD curve
\end{itemize}

\textbf{AS curve}

\begin{itemize}
    \item The inflation rate follows from the pricing decisions of producers
    \item The AS curve describes the change in prices (i.e., inflation)
    \item Prices $\to$ aggregate output of $Y$ produced by firms
    \item If firms produce a lot relative to some benchmark level of output ($\bar{Y}$), they tend to raise prices
    \[\pi = \bar{\pi} + \kappa(Y - \bar{Y}) \]
\end{itemize}

\textbf{Assessing the Economic Effects of the Coronavirus with the AS-AD Model}

Split the economy into 2 groups:

\begin{itemize}
    \item directly affected sectors: businesses that are directly affected by containment measures (e.g., restaurants)
    \item non-affected sectors: all other businesses (e.g., universities can switch to online education, insurances to home office)
\end{itemize}


\section{New Keynesian IS Curve}

\textbf{Household utility maximization problem:}
\begin{align*}
    & \max_{\{C_{t+j},B_{t+j},N_{t+j} \}_{j=0}^{\infty}} \quad 
    \mathbb{E}_t\left[
    \sum_{j=0}^{\infty}\beta^j Z_{t+j}
    \left(
    \frac{ C_{t+j}^{1-\sigma} }{1-\sigma} -
    \frac{ N_{t+j}^{1+\phi} }{1+\phi} 
    \right) \right] \\
    & \textrm{s.t.} \quad 
    P_{t+j}C_{t+j}+Q_{t+j}B_{t+j}\leq B_{t-1+j}+W_{t+j}N_{t+j}+D_{t+j} 
\end{align*}

\begin{itemize}
    \item $Z_t$: preference shifter
    \item $Q_t$: Bond price in $t$, pays 1 in $t+1$
    \item $B_t$: quantity of bonds purchased in $t$
    \item $N_t$: labor supply
    \item $W_t$: hourly wage payment
    \item $P_t$: the price of a consumption bundle $C_t$
    \item $D_t$: dividends paid by firms to households
\end{itemize}

Optimal consumption:
\begin{equation*}
    C_t^{-\sigma}=\mathbb{E}_t\left[
    \beta(1+i_t)\frac{Z_{t+1}}{Z_t}\frac{1}{1+\pi_{t+1}}C_{t+1}^{-\sigma} 
    \right] \quad
    \textrm{Euler equation}
\end{equation*}

\begin{itemize}
    \item $\beta$ low: impatient, higher consumption today.
    \item $\frac{Z_{t+1}}{Z_t}$ high: assign a lot to today's utility relative to tomorrow.
    \item $i_t$ high: less consumption and more saving today.
    \item $\pi_{t+1}$ high: tomorrow's prices high, more consumption today.
    \item $\sigma$ low: households substitute consumption today more strongly with consumption tomorrow when the interest rate is raised ($\sigma$ equals the inverse elasticity of substitution).
\end{itemize}

Optimal labor:
\begin{equation*}
    \frac{N_t^{\phi}}{C_t^{-\sigma}}=\frac{W_t}{P_t}
\end{equation*}

Log-linear form of Euler equation (using first-order Taylor approximation and defining $\log\beta\equiv\rho$):
\begin{equation*}
    c_t\simeq \mathbb{E}_tc_{t+1} - \frac{1}{\sigma}(i_t-\mathbb{E}_t\pi_{t+1}-\rho)+\frac{1-\rho_z}{\sigma}z_t \quad
    \textrm{NKIS}
\end{equation*}

\section{New Keynesian Phillips Curve}

%\textbf{Firm profit maximization problem:}

Optimal allocation of spending across varieties:
\begin{align*}
    \max_{C_t(i)} & \quad C_t=\left(\int_{i=0}^1 C_t(i)^{\frac{\varepsilon-1}{\varepsilon}} \textrm{d}i \right)^{\frac{\varepsilon}{\varepsilon-1}} \\
    \textrm{s.t.} & \quad 
    \int_{i=0}^{1}P_t(i)C_t(i)\textrm{d}i\leq S_t
\end{align*}
Optimal consumption of good $i$:
\begin{equation*}
    C_t(i)=C_t\left(\frac{P_t(i)}{P_t} \right)^{-\varepsilon}
\end{equation*}
where
\begin{equation*}
    P_t\equiv\left(\int_{i=0}^{1}P_t(i)^{1-\varepsilon}\textrm{d}i \right)^{\frac{1}{1-\varepsilon}}
\end{equation*}

Good market clearing: $Y_t(i)=C_t(i)$, $Y_t=C_t$
\begin{equation*}
    Y_t(i)=Y_t\left(\frac{P_t(i)}{P_t} \right)^{-\varepsilon}
    \quad \textrm{Demand function}
\end{equation*}

\begin{equation*}
    Y_t(i)=A_tL_t(i)
    \quad \textrm{Production function}
\end{equation*}

\textbf{Profit maximization under flexible prices:}
\begin{align*}
    \max_{P_t(i)} & \quad \Pi_t(i)=P_t(i)Y_t(i) - W_t L_t(i) \\
     \max_{P_t(i)} & \quad P_t(i)Y_t\left(\frac{P_t(i)}{P_t} \right)^{-\varepsilon} -
     \frac{W_t}{A_t}Y_t\left(\frac{P_t(i)}{P_t} \right)^{-\varepsilon}
\end{align*}
Profit-maximizing price:
\begin{equation*}
    P_t^*(i)=\frac{\varepsilon}{\varepsilon-1}\frac{W_t}{A_t}
\end{equation*}
Aggregate production under flexible prices:
\begin{equation*}
    Y_t=A_tL_t
\end{equation*}

\textbf{Labor market equilibrium:} $L_t=N_t$, $Y_t=C_t$

\begin{equation*}
    \begin{cases}
        \dfrac{N_t^{\varphi}}{C_t^{-\sigma}}=\dfrac{W_t}{P_t} \\
        P_t^*(i)=\dfrac{\varepsilon}{\varepsilon-1}\dfrac{W_t}{A_t} \\
        Y_t=A_tL_t
    \end{cases}\implies
    \begin{cases}
        L_t=\left(A_t^{1-\sigma}\dfrac{\varepsilon-1}{\varepsilon} \right)^{\frac{1}{\varphi+\sigma}} \\
        Y_t=A_t^{\frac{1+\varphi}{\varphi+\sigma}}\left(\dfrac{\varepsilon-1}{\varepsilon} \right)^{\frac{1}{\varphi+\sigma}}
    \end{cases}
\end{equation*}
\textbf{Key insight:} Equilibrium production does not depend on prices.

\textbf{Profit maximization under sticky prices:}
\begin{equation*}
    \max_{P_i(i)} \quad PV=\mathbb{E}_t\left[
    \sum_{j=0}^{\infty}\frac{\theta^j}{R_{t+j}}
    \left(
    P_t(i)Y_{t+j}\left(\frac{P_t(i)}{P_{t+j}} \right)^{-\varepsilon} -
    \frac{W_{t+j}}{A_{t+j}}Y_{t+j}\left(\frac{P_t(i)}{P_{t+j}} \right)^{-\varepsilon}
    \right)
    \right]
\end{equation*}
where
\begin{equation*}
    R_{t+j}=\prod_{\tau=1}^{j}(1+i_{t+\tau}-\mathbb{E}_t\pi_{t+\tau})
\end{equation*}
Optimal price under sticky prices:
\begin{equation*}
    O_t=\mathbb{E}_t\left[\sum_{j=0}^{\infty}\omega_jP_{t+j}^* \right]
\end{equation*}
where
\begin{equation*}
    \omega_j=\frac{\frac{\theta^j}{R_{t+j}}Y_{t+j}P_{t+j}^{\varepsilon}}{\sum_{j=0}^{\infty}\frac{\theta^j}{R_{t+j}}Y_{t+j}P_{t+j}^{\varepsilon}}
\end{equation*}
\begin{itemize}
    \item $\theta^j$: probability that price is still in place in $t+j$
    \item $\frac{1}{R_{t+j}}$: the present value of the profit that is generated in $t+j$
    \item $Y_{t+j}P_{t+j}^{\varepsilon}$: how many units a firm can sell for a given price in $t+j$
\end{itemize}


Log-linear approximation of the optimal price:
\begin{equation*}
    o_t\simeq\sum_{j=0}^{\infty}\frac{(\beta\theta)^j}{\sum_{k=0}^{\infty}(\beta\theta)^k}\mathbb{E}_tp_{t+j}^*
\end{equation*}

\begin{equation*}
    \frac{\frac{\theta^j}{R_{t+j}}Y_{t+j}P_{t+j}^{\varepsilon}}{\sum_{j=0}^{\infty}\frac{\theta^j}{R_{t+j}}Y_{t+j}P_{t+j}^{\varepsilon}}
    \implies
    \frac{(\beta\theta)^j}{\sum_{k=0}^{\infty}(\beta\theta)^k}
\end{equation*}
\begin{itemize}
    \item In steady state $Y_{t+j}P_{t+j}^{\varepsilon}$ is constant and therefore cancels out
    \item In steady state we have $i=-\log\beta=\log\frac{1}{\beta}\simeq\frac{1}{\beta}-1$ and $\pi=0$, hence $1+i=\frac{1}{\beta}$
    \[R_{t+j}=\prod_{\tau=1}^{j}(1+i-\pi)=\prod_{\tau=1}^{j}\frac{1}{\beta}=\left(\frac{1}{\beta}\right)^j
    \Longleftrightarrow \frac{1}{R_{t+j}}=\beta^j \]
    \item $\theta^j$: probability that price is still in place after $j$ periods
    
\end{itemize}

\begin{align*}
    o_t&\simeq\sum_{j=0}^{\infty}\frac{(\beta\theta)^j}{\sum_{k=0}^{\infty}(\beta\theta)^k}\mathbb{E}_tp_{t+j}^* \\
    &=(1-\beta\theta) \sum_{k=0}^{\infty}(\beta\theta)^k\mathbb{E}_tp_{t+j}^* \\
    &=(1-\beta\theta)p_t^*+
    \underbrace{(1-\beta\theta) \sum_{k=1}^{\infty}(\beta\theta)^k\mathbb{E}_tp_{t+j}^*}_{\beta\theta\mathbb{E}_to_{t+1}} \\
    &=(1-\beta\theta)p_t^* + \beta\theta\mathbb{E}_to_{t+1}
\end{align*}
\begin{itemize}
    \item The optimal price today is a weighted average of today's flexible price and tomorrow's optimal price
    \item Higher discounting ($\beta\downarrow$) and more often firms adjust ($\theta\downarrow$), the larger weight on today's flexible price
\end{itemize}


\textbf{Aggregate price dynamics:}

The price index in $t$ is a weighted average between the price chosen by the adjusting firms ($o_t$) and last period's price index ($p_{t-1}$)
\begin{align*}
    p_t &\simeq (1-\theta)o_t + \theta p_{t-1} \\
    \underbrace{p_t - p_{t-1}}_{\pi_t} &\simeq (1-\theta)(o_t-p_{t-1}) \\
    \pi_t & \simeq (1-\theta)(o_t-p_{t-1})
\end{align*}
\textbf{Intuition:} inflation is driven by
\begin{itemize}
    \item $1-\theta$: how many adjusting firms reset prices
    \item $o_t-p_{t-1}$: how much adjusting firms change prices
\end{itemize}
Iterate above equation one period forward:
\begin{equation*}
    \mathbb{E}_t\pi_{t+1}=(1-\theta)(\mathbb{E}_to_{t+1}-p_t)
\end{equation*}

\begin{equation*}
    \begin{cases}
        o_t=(1-\beta\theta)p_t^* + \beta\theta\mathbb{E}_to_{t+1} \\
        \pi_t = (1-\theta)(o_t-p_{t-1}) \\
        \mathbb{E}_t\pi_{t+1}=(1-\theta)(\mathbb{E}_to_{t+1}-p_t)
    \end{cases}
    \implies
    \pi_t=\frac{1-\theta}{\theta}(1-\beta\theta)(p_t^*-p_t)+\beta\mathbb{E}_t\pi_{t+1}
\end{equation*}
Inflation is high whenever
\begin{itemize}
    \item expected inflation is high (comes from adjusting firms choosing high $o_t$ because they expect high $\mathbb{E}_t o_{t+1}$)
    \item current flexible prices are high relative to the price level ($p_t^*-p_t$) (comes from adjusting firms choosing high $o_t$ because $p_t^*$ is high)
\end{itemize}

\textbf{Labor market equilibrium:} $L_t=N_t$, $Y_t=C_t$
\begin{equation*}
    \begin{cases}
        \dfrac{N_t^{\varphi}}{C_t^{-\sigma}}=\dfrac{W_t}{P_t} \\
        P_t^*(i)=\dfrac{\varepsilon}{\varepsilon-1}\dfrac{W_t}{A_t}
    \end{cases}
    \implies
    \begin{cases}
        \dfrac{P_t^*}{P_t}=\dfrac{\varepsilon}{\varepsilon-1}A_t^{-\varphi-1}Y_t^{\varphi+\sigma} \\
        1=\dfrac{\varepsilon}{\varepsilon-1}A_t^{-\varphi-1}(Y_t^n)^{\varphi+\sigma}
    \end{cases}
    \implies
    \frac{P_t^*}{P_t}=\left(\frac{Y_t}{Y_t^n}\right)^{\varphi+\sigma}
\end{equation*}
Log-linearize above equation:
\begin{equation*}
    p^*-p_t=(\varphi+\sigma)\underbrace{(y_t-y_t^n)}_{\widetilde{y}_t}
\end{equation*}

\begin{equation*}
    \pi_t=\kappa\widetilde{y}_t + \beta\mathbb{E}_t\pi_{t+1}\quad \textrm{NKPK}
\end{equation*}
where
\begin{equation*}
    \kappa=\frac{1-\theta}{\theta}(1-\beta\theta)(\varphi+\sigma)
\end{equation*}


\end{document}
