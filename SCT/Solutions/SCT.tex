\documentclass[a4paper]{article}
\usepackage[utf8]{inputenc}
\usepackage{amsmath,amsfonts,amssymb,amsthm, mathrsfs}
%\usepackage{mathtools}
\usepackage{graphicx}
\usepackage{setspace}
\usepackage{comment}
\usepackage{multirow}
% \usepackage[a4paper, total={6in, 8in}]{geometry}
\usepackage[top=2.0cm, left=2.0cm, right=2.0cm, bottom=3.0cm]{geometry}
\usepackage[colorlinks = true, linkcolor = cyan, urlcolor  = cyan, citecolor = cyan, anchorcolor = cyan]{hyperref}

%\DeclareSymbolFont{letters}{OT1}{cmtt}{m}{n}
%\renewcommand{\familydefault}{\sfdefault}


\title{%
%\vspace{1em}
    \textsc{Exercises} \\
    \vspace{1mm}
    \large \textsc{Social Choice Theory}
}
\author{Wenjie Tu}
\date{Spring Semester 2022}

\setlength{\parindent}{0pt}
\setlength{\parskip}{1em}
%\onehalfspacing
\begin{document}

\maketitle

\section*{Part 1: Preliminaries}

\subsection*{Exercise 1.1}

\subsubsection*{(a)}

Let $X=\{x, y, z\}$. A binary relation $R$ on the set $X$ is as follows:
\[R=\{(x,y), (y,z), (z,y) \} \]

The asymmetric part is:
\[P=\{(x,y)\} \]

\begin{itemize}
    \item The induced asymmetric part $P=\{(x,y)\}$ is transitive, hence $R$ is quasi-transitive.
    \item $(x,y)\in R$ and $(y,z)\in R$ while $(x,z)\notin R$, a contradiction to transitivity.
\end{itemize}

Therefore $R=\{(x,y), (y,z), (z,y) \}$ is quasi-transitive but not transitive.

\subsubsection*{(b)}

Let $X=\{x, y, z\}$. A binary relation $R$ on the set $X$ is as follows:
\[R=\{(x,y), (y,z) \} \]

Its asymmetric part $P$ is:
\[P=\{(x,y), (y,z) \} \]

\begin{itemize}
    \item Since $(x,z)\notin P$, $P$ is not transitive and $R$ is not quasi-transitive.
    \item Since $(z,x)\notin P$, $P$ has no cycle and $R$ is acyclical.
\end{itemize}


\subsubsection*{(c)}

Need to show:
\[xIy \land yIz \implies xIz \]

Proof:

We can rewrite induced $I$ into binary relation:

\[xIy\implies [xRy \land yRx] \]
\[yIz\implies [yRz \land zRy] \]

\[xRy \land yRz \implies xRz \quad \text{by transitivity} \]
\[yRx \land zRy \implies zRx \quad \text{by transitivity} \]

\[xRz \land zRx \implies xIz \]


\subsection*{Exercise 1.2}

\subsubsection*{(a)}

$\succeq$ is a preference

\[x\sim y \]
\[y\succ z \]
\[z\succ w \]

\[G(\{w,x,y,z\})=\{x,y\} \]
\[M(\{w,x,y,z\})=\{x,y\} \]

\[R=\{(x,y), (y,x), (y,z), (z,w) \} \]

\subsubsection*{(b)}

\[M(\{x,y,z\}, R)=\{\}=G(\{x,y,z\}) \]

\subsection*{Exercise 1.3}

\subsubsection*{(a)}

We know that $x$ is chosen in the larger set (i.e. $C(\{x,y,z\})=\{x\}$), and we need to check whether $x$ is still chosen from all the possible subset containing $x$.

\[C(\{x,y\})=\{x\} \]
\[C(\{x,z\})=\{x\} \]

Hence, property $\alpha$ is satisfied.

We cannot find any two alternatives that are generate by the choice function. Hence, we cannot find a counter example that violates property $\beta$ (i.e. $\beta$ is satisfied).

\subsubsection*{(b)}

We know that $z$ is chosen from the larger set (i.e. $C(\{x,y,z\})=\{z\}$), and we need to check whether $z$ is still chosen from all the possible subset containing $z$.

Base relation:

\[C(\{x,z\})=\{x,z\} \]
\[C(\{y,z\})=\{z\} \]

Preference:
\[x\sim z \succ y \]

Hence, property $\alpha$ is satisfied.

We see that $\{x,z\}$ is chosen from the smaller set $\{x,z\}$ (i.e. $C(\{x,z\})=\{x,z\}$), and $z$ is chosen from the larger set $\{x,y,z\}$ while $x$ is not, hence a violation of property $\beta$.

\subsubsection*{(c)}

We know that $\{y,z\}$ is chosen from the larger set $\{x,y,z\}$, and we need to check whether $\{y,z\}$ is still chosen from all the possible subset containing $\{y,z\}$.

We see:
\[C(\{y,z\})=\{y,z\} \]

Hence, property $\alpha$ is satisfied.

We see that $\{y,z\}$ is chosen from the smaller set $\{y,z\}$, and $\{y,z\}$ is chosen from the larger set containing both $y$ and $z$. Hence, property $\beta$ is satisfied.


\subsection*{Exercise 1.4}

Shorthands for choices:
\begin{itemize}
    \item $P$: peatnuts
    \item $A$: apple juice
    \item $M$: mineral water
    \item $B$: beer
\end{itemize}

\subsubsection*{(a)}

\[X=\{P,A,M,B \} \]

\[C(\{P,A,M\})=\{P, A\} \]

\[C(\{P,A,M,B\})=\{P, B\} \]

$\beta$ is violated?

Better translation of the information:
\[X=\{PA, PM, PB \} \]

\[C(\{PA,PM\})=\{PA\} \]

\[C(\{PA,PM,PB\})=\{PB\} \]

$\beta$ is satisfied.

\section*{Part 2: The Problem of Social Choice}

\subsection*{Exercise 2.1}

\subsubsection*{(a)}

\begin{table}[!htbp]
    \centering
    \begin{tabular}{c|c|}
        \# & preferences         \\ 
        \hline
        1  & $x\: P\: y\: P\: z$ \\
        1  & $y\: P\: z\: P\: x$ \\
        1  & $z\: P\: x\: P\: y$ \\
        \hline
    \end{tabular}
\end{table}

\begin{table}[!htbp]
    \centering
    \begin{tabular}{ccc}
        \hline
        Pairwise comparison & Votes               & Social preference  \\ 
        \hline
        $x\: \text{vs.} y$  & $2\: \text{vs.} 1$  & $xPy$              \\
        $x\: \text{vs.} z$  & $1\: \text{vs.} 2$  & $zPx$              \\
        $y\: \text{vs.} z$  & $2\: \text{vs.} 1$  & $yPz$              \\
        \hline
    \end{tabular}
\end{table}

This leads to an outcome that is not transitive and not acyclical.

There is no Condorcet winner (loser) as each alternative loses in the pairwise voting at once.

\subsubsection*{(b)}

Fix any $R^*\in R$, let $\mathscr{A}=\{(R^*, R^*,\cdots, R^*) \}$. In other words, all individuals share the same preference. In this example, the pairwise majority voting is an SWF.

\subsection*{Exercise 2.2}

When $m=2$, PV, IR, PM, CO, and BC are identical. They all collapse to the majority voting method.

\subsection*{Exercise 2.3}

\begin{table}[!htbp]
    \centering
    \begin{tabular}{c|c|}
        \# & preferences         \\ 
        \hline
        2  & $x\: P\: w\: P\: y\: P\: z$ \\
        2  & $y\: P\: w\: P\: z\: P\: x$ \\
        1  & $w\: P\: z\: P\: x\: P\: y$ \\
        \hline
    \end{tabular}
\end{table}

\subsubsection*{PV:}

In plurality voting, we only consider the top-ranked alternatives given by voters.
\[x\: I\: y\: P\: w\: P\: z  \]

\subsubsection*{IR:}

\begin{itemize}
    \item Stage 1: $z$ is eliminated
    \item Stage 2: $w$ is eliminated
    \item Stage 3: $x$ wins with 3 votes (majority)
\end{itemize}

\subsubsection*{PM:}


\begin{table}[!htbp]
    \centering
    \begin{tabular}{ccc}
        pairwise comparison & votes  & binary relation \\
        \hline 
        $w:x$  & $3:2$  & $wPx$           \\
        $w:y$  & $3:2$  & $wPy$           \\
        $w:z$  & $5:0$  & $wPz$           \\
        $x:y$  & $3:2$  & $xPy$           \\
        $x:z$  & $2:3$  & $zPx$           \\
        $y:z$  & $4:1$  & $yPz$           \\
        \hline
    \end{tabular}
\end{table}

The pairwise majority voting method delivers a Condorcet winner $w$ but there exists a cycle in $x, y, z$.

\subsubsection*{CO:}

\begin{table}[!htbp]
    \centering
    \begin{tabular}{c|c|c|c|c|c}
            & $w$  & $x$  & $y$  & $z$  & $\sum$ \\
        \hline 
        $w$ &      & $+1$ & $+1$ & $+1$ & $3$    \\
        \hline 
        $x$ & $-1$ &      & $+1$ & $-1$ & $-1$   \\
        \hline
        $y$ & $-1$ & $-1$ &      & $+1$ & $-1$   \\
        \hline
        $x$ & $-1$ & $+1$ & $-1$ &      & $-1$   \\
        \hline
    \end{tabular}
\end{table}

Copeland method delivers the social preference $w \: P \: x \: I \: y \: I \: z$.

\subsubsection*{BC:}

\begin{table}[!htbp]
    \centering
    \begin{tabular}{c|c|cccc|}
        \# & preferences                  & $w$  & $x$  & $y$  & $z$ \\ 
        \hline
        2  & $x\: P\: w\: P\: y\: P\: z$  & $2$  & $3$  & $1$  & $0$ \\
        2  & $y\: P\: w\: P\: z\: P\: x$  & $2$  & $0$  & $3$  & $1$ \\
        1  & $w\: P\: z\: P\: x\: P\: y$  & $3$  & $1$  & $0$  & $2$ \\
        \hline
           &                              & $11$ & $7$  & $8$  & $4$ \\
        \hline
    \end{tabular}
\end{table}

Borda Count method delivers the social preference $wPyPxPz$

\subsubsection*{EM:}

There exists at least one individual who ranks $x, y, w$ as his/her top preference, hence the Pareto efficient set is $X^E=\{w, x, y\}$. All individuals strictly prefer $w$ over $z$, hence $z$ is strictly dominated by $w$. $X^I=\{z\}$.

Pareto Efficient Method delivers the social preference $wIxIyPz$.

\subsubsection*{PE:}

\begin{table}[!htbp]
    \centering
    \begin{tabular}{cc}
        pairwise comparison & Pareto extension rule \\
        \hline 
        $w:x$  & $wIx$           \\
        $w:y$  & $wIy$           \\
        $w:z$  & $wPz$           \\
        $x:y$  & $xIy$           \\
        $x:z$  & $zIx$           \\
        $y:z$  & $yIz$           \\
        \hline
    \end{tabular}
\end{table}

Pareto Extension Rule delivers an outcome that violates transitivity but is quasi-transitive. This is SDF but not SWF.

\subsection*{Exercise 2.4}

\subsubsection*{(a)}

\begin{table}[!htbp]
    \centering
    \begin{tabular}{c|c|}
        \# & preferences         \\ 
        \hline
        2  & $x\: P\: z\: P\: y$ \\
        2  & $y\: P\: z\: P\: x$ \\
        1  & $z\: P\: y\: P\: x$ \\
        \hline
    \end{tabular}
\end{table}

Applying IR:
\begin{itemize}
    \item Stage 1: $z$ is eliminated.
    \item Stage 2: $y$ wins with 3 votes (majority).
\end{itemize}

\begin{table}[!htbp]
    \centering
    \begin{tabular}{ccc}
        pairwise comparison & votes  & binary relation \\
        \hline 
        $x:y$  & $2:3$  & $yPx$           \\
        $x:z$  & $2:3$  & $zPx$           \\
        $y:z$  & $2:3$  & $zPy$           \\
        \hline
    \end{tabular}
\end{table}

\begin{itemize}
    \item $z$ wins all pairwise comparison hence a Condorcet winner.
    \item IR is not a Condorcet method.
\end{itemize}

\subsubsection*{(b)}

Proof by contradiction:

No, a Condorcet winner is Pareto efficient, otherwise it would lose at least one pairwise vote.

\subsubsection*{(c)}

We can construct the following example resulting $x$ to be a Condorcet winner:

\begin{table}[!htbp]
    \centering
    \begin{tabular}{c|c|}
        \# & preferences         \\ 
        \hline
        2  & $x\: P\: y$ \\
        1  & $y\: P\: x$ \\
        \hline
    \end{tabular}
\end{table}

There is no Pareto dominance between $x$ and $y$ so EM delivers social indifference between $x$ and $y$ (i.e. $xIy$).

Same argument for PE.

\subsection*{Exercise 2.5}

\subsubsection*{(a)}

For arbitrary values of $m$:
\begin{itemize}
    \item Anti-plurality voting and rejection voting are equivalent.
    \item Nameless example I and nameless example II are equivalent.
\end{itemize}

For $m=3$:
\begin{itemize}
    \item Anti-plurality voting and rejection voting are equivalent.
    \item Borda count, nameless example I, and nameless example II are equivalent.
\end{itemize}

\subsubsection*{(b)}

\[s=(m, m-1, \cdots, 1) \]

\subsubsection*{(c)}

\[s=(m^2, (m-1)^2, \cdots, 1^2) \]

\subsection*{Exercise 2.6}

\begin{table}[!htbp]
    \centering
    \begin{tabular}{c|c|cccc|}
        \# & preferences                  & $v$  & $x$  & $y$  & $z$ \\ 
        \hline
        1  & $x\: P\: z\: P\: v\: P\: y$  & $1$  & $3$  & $0$  & $2$ \\
        1  & $y\: P\: z\: P\: v\: P\: x$  & $1$  & $0$  & $3$  & $2$ \\
        1  & $v\: P\: z\: P\: y\: P\: x$  & $3$  & $0$  & $1$  & $2$ \\
        1  & $x\: P\: y\: P\: v\: P\: z$  & $1$  & $3$  & $2$  & $0$ \\
        \hline
           &                              & $6$ & $6$  & $6$  & $6$ \\
        \hline
    \end{tabular}
\end{table}

$s^1$ delivers social preference $vIxIyIz$

\begin{table}[!htbp]
    \centering
    \begin{tabular}{c|c|cccc|}
        \# & preferences                  & $v$  & $x$  & $y$  & $z$ \\ 
        \hline
        1  & $x\: P\: z\: P\: v\: P\: y$  & $1/4$  & $1$  & $0$  & $3/4$ \\
        1  & $y\: P\: z\: P\: v\: P\: x$  & $1/4$  & $0$  & $1$  & $3/4$ \\
        1  & $v\: P\: z\: P\: y\: P\: x$  & $1$  & $0$  & $1/4$  & $3/4$ \\
        1  & $x\: P\: y\: P\: v\: P\: z$  & $1/4$  & $1$  & $3/4$  & $0$ \\
        \hline
           &                              & $7/4$ & $8/4$  & $8/4$  & $9/4$ \\
        \hline
    \end{tabular}
\end{table}

$s^2$ delivers social preference $zPxIyPv$

\begin{table}[!htbp]
    \centering
    \begin{tabular}{c|c|cccc|}
        \# & preferences                  & $v$  & $x$  & $y$  & $z$ \\ 
        \hline
        1  & $x\: P\: z\: P\: v\: P\: y$  & $1/4$  & $1$  & $0$  & $1/2$ \\
        1  & $y\: P\: z\: P\: v\: P\: x$  & $1/4$  & $0$  & $1$  & $1/2$ \\
        1  & $v\: P\: z\: P\: y\: P\: x$  & $1$  & $0$  & $1/4$  & $1/2$ \\
        1  & $x\: P\: y\: P\: v\: P\: z$  & $1/4$  & $1$  & $1/2$  & $0$ \\
        \hline
           &                              & $7/4$ & $8/4$  & $7/4$  & $6/4$ \\
        \hline
    \end{tabular}
\end{table}

$s^3$ delivers social preference $xPvIyPz$

\section*{Part 3: Arrow's Theorem}

\subsection*{Exercise 3.1}

\subsubsection*{(a) Plurality voting (PV)}

\underline{[I] Independence of Irrelevant Alternatives Axiom is violated. }

The counter example is given as follows:

\begin{table}[!htbp]
    \centering
    \begin{tabular}{c|c|}
        $\mathbf{R}$ & preferences         \\ 
        \hline
        2  & $x\: P\: y\: P\: z$ \\
        1  & $y\: P\: x\: P\: z$ \\
        1  & $x\: P\: z\: P\: y$ \\
        \hline 
        $f^{PV}$ & $x\: P\: y\: P\: z$
    \end{tabular}
    \qquad $\to$ \qquad
    \centering
    \begin{tabular}{c|c|}
        $\mathbf{R'}$ & preferences         \\ 
        \hline
        2  & $z\: P\: x\: P\: y$ \\
        1  & $y\: P\: x\: P\: z$ \\
        1  & $x\: P\: z\: P\: y$ \\
        \hline 
        $f^{PV}$ & $z\: P\: y\: I\: x$
    \end{tabular}
\end{table}

Note: $z$ is the irrelevant alternative. We change the preferences for $z$ for the first two people while maintaining the relative positions for $x$ and $y$ (i.e. $xPy$). From $\mathbf{R}\to\mathbf{R'}$, we see social preference between $x$ and $y$ changes, hence a violation of [I].

\underline{[P] Weak Pareto Principle is violated. }

The counter example is given as follows:

\begin{table}[!htbp]
    \centering
    \begin{tabular}{c|c|}
        $\mathbf{R}$ & preferences         \\ 
        \hline
        2  & $z\: P\: x\: P\: y$ \\
        \hline 
        $f^{PV}$ & $z\: P\: x\: I\: y$
    \end{tabular}
\end{table}

\underline{[U] Universality is satisfied. }

\begin{itemize}
    \item Plurality voting (PV) is well-defined method and always applicable as there are no constraints in definition/preferences.
    \item The resulting outcomes are preferences as plurality voting method is a ranking method based on numbers/votes. Numbers are transitive, hence outcomes are transitive (PV is a SWF).
\end{itemize}

\underline{[D] Non-dictatorship is satisfied.} 

Each individual has only one vote. In order to impose the own preference on the whole society, the dictator must have ``majority'' votes, which is not possible for plurality voting where each individual has one single vote.

Or proof by contradiction.

\subsubsection*{(b) Pairwise majority voting (PM)}

\underline{[I] Independence of Irrelevant Alternatives Axiom is satisfied. }

Pairwise majority voting method is a pairwise method by construction. To determine the ranking between $x$ and $y$, we do pairwise comparison. If preferences between $x$ and $y$ do not change, the outcome does not change.

\underline{[D] Non-dictatorship is satisfied.} 

As long as the majority is against you, you cannot enforce your own preference on the society. This holds for everybody. Nobody can be a dictator.

\underline{[P] Weak Pareto Principle is satisfied.}

If everybody has a strict preference for $x$ over $y$, the outcome will be an unanimous vote for $x$ over $y$ as we only do pairwise comparison, hence a strict preference for $x$ over $y$ for the society.

\underline{[U] Universality is violated.}

A counter example is constructed as follows:

\begin{table}[!htbp]
    \centering
    \begin{tabular}{c|c|}
        \# & preferences         \\ 
        \hline
        1  & $x\: P\: y\: P\: z$ \\
        1  & $y\: P\: z\: P\: x$ \\
        1  & $z\: P\: x\: P\: y$ \\
        \hline
    \end{tabular}
\end{table}

The example given above leads to an acyclical preferences. This is not a SWF on the full domain. We can easily get violations of acyclicity and transitivity.

\subsubsection*{(c) Copeland method (CO)}

\underline{[U] Universality is satisfied.}

Copeland method by definition is a number-based voting method and hence it is always applicable as numbers are always transitive.

\underline{[D] Non-dictatorship is satisfied.} 

As long as the majority is against you, you cannot enforce your own preference on the society. This holds for everybody. Nobody can be a dictator.

\underline{[P] Weak Pareto Principle is satisfied.}

Suppose $x$ strictly Pareto dominates $y$, under Copeland method $y$ loses every pairwise vote that $x$ loses. In other words, whenever $x$ loses, $y$ loses as well. 
\[CO(x, \mathbf{R}) \geq CO(y,\mathbf{R})+2 \]

\underline{[I] Independence of Irrelevant Alternatives Axiom is violated.}

A counter example is constructed as follows:

\begin{table}[!htbp]
    \centering
    \begin{tabular}{c|c|}
        $\mathbf{R}$ & preferences         \\ 
        \hline
        1  & $x\: P\: y\: P\: z$ \\
        1  & $y\: P\: x\: P\: z$ \\
        \hline 
        $f^{CO}$ & $x\: I\: y\: P\: z$
    \end{tabular}
    \qquad $\to$ \qquad
    \centering
    \begin{tabular}{c|c|}
        $\mathbf{R'}$ & preferences         \\ 
        \hline
        1  & $x\: P\: y\: P\: z$ \\
        1  & $y\: P\: z\: P\: x$ \\
        \hline 
        $f^{CO}$ & $y\: P\: x\: P\: z$
    \end{tabular}
\end{table}

Note: $z$ is the irrelevant alternative. We change the preference for $z$ for the second person while maintaining the relative positions for $x$ and $y$ (i.e. $xPy$). From $\mathbf{R}\to\mathbf{R'}$, we see social preference between $x$ and $y$ changes, hence a violation of [I].

\subsubsection*{(d) Borda count (BC)}

\underline{[D] Non-dictatorship is satisfied.} 

Borda count method is formulated in a linear way for scoring vector and all votes use the same scoring vector, hence no dictatorship possible.

\underline{[U] Universality is satisfied.}

Borda count is a number-based voting method so it always leads to transitive preferences. It is applicable to all preference profiles as there are no constraints in definition.

\underline{[P] Weak Pareto Principle is satisfied.}

By definition, if $xP_i y\:\forall i$, 
\[BC(x, \mathbf{R})\geq BC(x, \mathbf{R}) + n \]
Hence, the society strictly prefers $x$ over $y$.

\underline{[I] Independence of Irrelevant Alternatives Axiom is violated.}

A counter example is constructed as follows:

\begin{table}[!htbp]
    \centering
    \begin{tabular}{c|c|}
        $\mathbf{R}$ & preferences         \\ 
        \hline
        1  & $x\: P\: y\: P\: z$ \\
        1  & $y\: P\: x\: P\: z$ \\
        \hline 
        $f^{BC}$ & $x\: I\: y\: P\: z$
    \end{tabular}
    \qquad $\to$ \qquad
    \centering
    \begin{tabular}{c|c|}
        $\mathbf{R'}$ & preferences         \\ 
        \hline
        1  & $x\: P\: y\: P\: z$ \\
        1  & $y\: P\: z\: P\: x$ \\
        \hline 
        $f^{BC}$ & $y\: P\: x\: P\: z$
    \end{tabular}
\end{table}

Note: $z$ is the irrelevant alternative. We change the preference for $z$ for the second person while maintaining the relative positions for $x$ and $y$ (i.e. $xPy$). From $\mathbf{R}\to\mathbf{R'}$, we see social preference between $x$ and $y$ changes, hence a violation of [I].

\subsubsection*{(e) Pareto Efficiency Method (EM)}

\underline{[D] Non-dictatorship is satisfied.} 

Proof by contradiction: suppose there exists a dictator who strictly prefers $x$ over $y$, the society is supposed to strictly prefer $x$ over $y$. Now there is an individual who strictly prefers $y$ over $x$, by definition of Pareto Efficiency Method, the society should be indifferent between $x$ and $y$ as there is no Pareto dominance between $x$ and $y$.

\underline{[P] Weak Pareto Principle is violated.}

\begin{table}[!htbp]
    \centering
    \begin{tabular}{c|c|}
        \# & preferences         \\ 
        \hline
        2  & $z\: P\: x\: P\: y$ \\
        \hline
        $f^{EM}$ & $z\: P\: x\: I\: y$
    \end{tabular}
\end{table}

\begin{itemize}
    \item $X^E=\{z \}$
    \item $X^I=\{x,y \}$
\end{itemize}

All individuals strictly prefer $x$ over $y$ but by applying Pareto Efficiency Method the society is indifferent between $x$ and $y$, hence a violation of Weak Pareto Principle.

\underline{[I] Independence of Irrelevant Alternatives Axiom is violated.}

A counter example is constructed as follows:

\begin{table}[!htbp]
    \centering
    \begin{tabular}{c|c|}
        $\mathbf{R}$ & preferences         \\ 
        \hline
        2  & $z\: P\: x\: P\: y$ \\
        \hline 
        $f^{EM}$ & $z\: P\: x\: I\: y$
    \end{tabular}
    \qquad $\to$ \qquad
    \centering
    \begin{tabular}{c|c|}
        $\mathbf{R'}$ & preferences         \\ 
        \hline
        2  & $x\: P\: y\: P\: z$ \\
        \hline 
        $f^{EM}$ & $x\: P\: y\: I\: z$
    \end{tabular}
\end{table}

Note: $z$ is the irrelevant alternative. We change the preferences for $z$ while maintaining the relative positions for $x$ and $y$ (i.e. $xPy$) for all individuals. From $\mathbf{R}\to\mathbf{R'}$, we see social preference between $x$ and $y$ changes, hence a violation of [I].

\underline{[U] Universality is satisfied.}

\begin{itemize}
    \item It is well-defined and always applicable.
    \item It always gives us preferences with at most two indifference classes.
\end{itemize}

\subsubsection*{(f) The preference of society coincides with the preference of individual $h\in N$}

\underline{[D] Non-dictatorship is violated.} 

By definition, the individual $h$ can always impose the own preference on the whole society and hence a dictator.

\underline{[U] Universality is satisfied.}

\begin{itemize}
    \item There are obviously no constraints built into the method and it is always applicable no matter how people's preferences look like.
    \item The dictator's preference is a preference. By definition of this method, the social preference will be a preference.
\end{itemize}

\underline{[P] Weak Pareto Principle is satisfied.}

If everybody strictly prefers $x$ over $y$, then the dictator also strictly prefers $x$ over $y$ because the dictator is one of those people.

\underline{[I] Independence of Irrelevant Alternatives Axiom is satisfied.}

Social preference is only determined by the dictator's preference. [I] holds for a single individual hence it also holds for the society with dictatorship. 

\subsubsection*{(g) Construct an example where all axioms but [P] are satisfied}

Fix any $R^*\in R$ and define $f(\mathbf{R})=R^*\:\forall\:\mathbf{R}\in\mathscr{R}^{U}$
\begin{itemize}
    \item [P] is violated
    \item [I] is satisfied
    \item [U] is satisfied. The outcome will be a preference and the method is well-defined on the universal domain
    \item [D] is satisfied. $R^*$ is arbitrary. Only the rule maker decides the dictator. Voters themselves do not know which one of them will be the dictator.
\end{itemize}

\subsection*{Exercise 3.2}

[$\mathrm{\bar{M}}$] is violated.

A counter example is constructed as follows:

\begin{table}[!htbp]
    \centering
    \begin{tabular}{c|c|}
        $\mathbf{R}$ & preferences         \\ 
        \hline
        2  & $x\: P\: z\: P\: y$ \\
        2  & $y\: P\: z\: P\: x$ \\
        1  & $z\: P\: x\: P\: y$ \\
        \hline 
           & $x$ wins
    \end{tabular}
    \qquad $\to$ \qquad
    \centering
    \begin{tabular}{c|c|}
        $\mathbf{R'}$ & preferences         \\ 
        \hline
        2  & $x\: P\: z\: P\: y$ \\
        2  & $z\: P\: y\: P\: x$ \\
        1  & $z\: P\: x\: P\: y$ \\
        \hline 
           & $z$ wins
    \end{tabular}
\end{table}

[$\mathrm{\bar{D}}$] is satisfied.

Proof by contradiction.

[$\mathrm{\bar{U}}$] is satisfied.

\begin{itemize}
    \item People are allowed to express whatever preferences they have. We have well-specified tie-breaking rules such that at the end there will always be a winner even if we go to the last stage.
    \item It is a well-specified method on the universal domain.
\end{itemize}

[$\mathrm{\bar{P}}$] is satisfied.

Suppose $yP_i x\:\forall i$
\begin{enumerate}
    \item As long as $y$ is around, $x$ gets zero votes
    \item $y$ can be eliminated only if it gets zero votes
    \item Updated votes do not change after elimination of alternatives with zero votes
    \item Either now or after some steps $x$ will be eliminated
    \item $x$ cannot be the winner
\end{enumerate}

\end{document}
