\documentclass[a4paper]{article}
\usepackage[utf8]{inputenc}
\usepackage{amsmath,amsfonts,amssymb,amsthm}
%\usepackage{mathtools}
\usepackage{graphicx}
\usepackage{setspace}
\usepackage{comment}
% \usepackage[a4paper, total={6in, 8in}]{geometry}
\usepackage[top=2.0cm, left=2.0cm, right=2.0cm, bottom=3.0cm]{geometry}
\renewcommand{\familydefault}{\sfdefault}

\title{%
     Global Poverty and Economic Development
}
\author{Wenjie Tu}
\date{Fall Semester 2021}

\setlength{\parindent}{0pt}
\setlength{\parskip}{1em}
%\onehalfspacing
\begin{document}

\maketitle

\section{Adverse Selection}
\textbf{Setup:}
\begin{itemize}
    \item Projects need startup cost $L$
    \item Entrepreneurs (borrowers) vary in their unobservable type: risky or safe
    \begin{itemize}
        \item Risky borrowers: invest in risky assets and obtain return $R'>L$ with probability $p$ and zero return with probability $1-p$
        \item Safe borrowers: invest in safe assets and always obtain return $R<R'$
        \item No borrower's action/effort
    \end{itemize}
    \item Only one potential borrower of each type and one lender who can issue only one single loan, $L$
    \begin{itemize}
        \item If both borrowers apply, the lender randomly picks one (the lender cannot observe the borrow's type)
    \end{itemize}
\end{itemize}
\textbf{Solution:}
\begin{itemize}
    \item Maximum interest rate that borrowers will accept
    \begin{itemize}
        \item Safe borrower: $i_s=\frac{R-L}{L}$
        \item Risky borrower: $i_r=\frac{R'-L}{L}$ (she pays only if the project succeeds)
    \end{itemize}
    \item If the lender offers a loan at interest rate $i_s$, both borrowers willy apply and the lender's expected profit is:
    \[\pi_s=\frac{1}{2}L(1+i_s)+\frac{1}{2}Lp(1+i_s)-L \]
    \item If the lender offers a loan at interest rate $i_r$, only the risky borrower will apply and the lender's expected profit is:
    \[\pi_r=pL(1+i_r)-L \]
    \item The lender will choose $i_s$ if $\pi_s>\pi_r$
    \[p<\frac{R}{2R'-R} \]
    \item Intuition
    \begin{itemize}
        \item By raising the interest rate, only risky borrowers apply (\underline{adverse selection}) $\to$ higher interest may reduce lender's profit
        \item If the lender chooses $i_s$, there is \textbf{credit rationing}: demand exceeds supply at $i_s$, but the lender does not raise the price
    \end{itemize}
\end{itemize}

\section{Moral Hazard}
\textbf{Setup:}
\begin{itemize}
    \item An entrepreneur can invest in a project that leads to return $R$ with probability $e$ and 0 otherwise
    \item The entrepreneur chooses the effort level $e$:
    \begin{itemize}
        \item Cost of effort: $c(e)=\frac{1}{2}ce^2$
    \end{itemize}
    \item Opportunity cost of capital: $\rho$
    \item Opportunity cost of labor/time: $u$
\end{itemize}
\textbf{Solution:}
\begin{itemize}
    \item If the entrepreneur can self-finance the project, her maximization problem is:
    \[\max_{e}\quad eR+(1-e)0-\frac{1}{2}ce^2-\rho-u \]
    \item The optimal (First Best) level of effort is:
    \[e^{FB}=\frac{R}{c} \]
    \begin{itemize}
        \item assume an interior solution $e<1$
    \end{itemize}
    \item Now assume the entrepreneur cannot self-finance: she has illiquid wealth $w$ that she can use as collateral for a loan
    \item The entrepreneur can get a loan from a lender:
    \begin{itemize}
        \item He pays back interest $r$ if the project  succeeds
        \item He pays collateral $w$ if the project does not succeed (extreme case of limited liability: $w=0$)
    \end{itemize}
    \item Entrepreneur (borrower) payoff:
    \[\pi^B=e(R-r)+(1-e)(-w)-\frac{1}{2}ce^2-u \]
    \item Lender payoff:
    \[\pi^L=er+(1-e)w-\rho \]
    \item If the two parts could contract on effort, they would choose the level that maximizes the joint surplus ($\pi^B+\pi^L$), which is again $e^{FB}$
    \item Now assume that the lender and the borrower cannot contract on effort
    \begin{itemize}
        \item Notice that the lender observes the type of the borrower but he still cannot contract on the action of the borrower
    \end{itemize}
    \item For a given interest, the borrower will choose the level of effort that maximizes $\pi^B$ (\textit{Incentive Compatibility Constraint})
    \[e^{SB}=\frac{R-r+w}{c} \]
    \begin{itemize}
        \item If $w<r$, then $e^{SB}<e^{FB}$. Why?
    \end{itemize}
    \item Assume perfect competition among lenders $\to$ Lender's expected profit must equal the cost of capital (\textit{Zero Profit Condition}):
    \[er+(1-e)w=\rho \]
    \item Blug the IC into the ZPC, we obtain
    \[ce^2-eR+(\rho-w)=0 \]
    \item The solution is the larger root:
    \[e^*(w)=\frac{R+\sqrt{R^2-4c(\rho-w)}}{2c} \]
    \begin{itemize}
        \item The lender is indifferent between two roots, but the borrower is better off with the larger root
        \item $e^*$ is increasing in $w$. If $w=\rho$, $e^*=e^{FB}$
    \end{itemize}
    \item We can also solve for the equilibrium interest (i.e., loan$\times$(1+interest rate))
    \[r^*(w)=w+\frac{R-\sqrt{R^2-4c(\rho-w)}}{2} \]
    \item It can be shown that, for $w<\rho$, $\frac{\partial r^*(w)}{\partial w}<0$
    \begin{itemize}
        \item Richer borrowers get the loan at a lower interest rate and in equilibrium they will be more successful in their projects
        \item If $w$ is very low, it may be impossible to satisfy the lenders' ZPC while also ensuring the borrower's utility is above $u$ $\to$ poor borrowers do not receive the loan
    \end{itemize}
\end{itemize}

\textbf{Example:}

\textbf{Question 1:} Agents can undertake a project at cost of 1. The project has outcome $y$ if it succeeds and 0 otherwise. The probability of success is equal to the amount of effort $e$ the agent exerts (or probability = 1 if $e>1$). The cost of effort is $\frac{1}{2}ce^2$
\begin{enumerate}
    \item What is the first best effort choice? In this and next questions, assume $c\geq y$.
    \item Suppose the agent cannot self-finance the project, but she has to borrow from a bank at (gross) interest rate $r>1$. Assume the agent has limited liability. Write down the borrower’s problem. What is the level of effort chosen by the borrower? How does it compare to the first best? Why?
    \item Now suppose two borrowers $i$, $j$ (with same $c$) are in a group lending scheme: if agent $i$ succeeds but agent $j$ fails, agent i pays a cost $k$ to the lender (assume $k<c$). Suppose the two borrowers choose independently their level of effort, taking as given the choice of the other borrower. What is the (symmetric) level of effort the borrowers choose?
\end{enumerate}



\textbf{Question 2:} An entrepreneur can invest k in a project and obtain $F(k)$. He was own wealth $w<k$ and need to borrow the rest at interest rate r. When the time to repay the loan comes, the entrepreneur can run away by paying a cost $\eta$ per unit of capital. In other words, the lender cannot enforce repayment.
\begin{enumerate}
    \item When will the borrower choose to default? Therefore, what is the maximum amount a lender will lend?
    \item What is the relationship between the amount invested and wealth?
\end{enumerate}
Now suppose that the borrower’s cost of defaulting is zero unless the lender bears a monitoring cost $\phi$ (in which case the cost of defaulting is again $\eta$ per unit of $k$). Also, suppose that the cost of capital is $\rho$. The equilibrium in the lending market is driven by a zero profit condition for the lender that equates the profits lender makes on loan to the cost of capital.
\begin{enumerate}
    \setcounter{enumi}{2}
    \item What is zero profit condition for the lender?
    \item What is the maximum loan amount a borrower can get? (hint: equate the lender zero profit condition and the incentive constraint for the borrower)
    \item What is the interest rate when the credit constraint binds? How does the interest rate compare to the cost of capital $\rho$? How does this comparison depend on the monitoring cost $\phi$?
\end{enumerate}

\section{Quasi-Hyperbolic Discounting vs. EU}
\begin{equation*}
    U^t(c_t,c_{t+1},\cdots,c_T)=\delta^{t-1}u(c_t)+\beta\sum_{\tau=t+1}^{T}\delta^{\tau-1}u(c_{\tau})
\end{equation*}
\begin{equation*}
    \delta_{t,s}=
    \begin{cases}
    1, & \textrm{if } t=s \\
    \beta\delta^{t-s}, & \textrm{if } t>s
    \end{cases}
\end{equation*}
\begin{itemize}
    \item $\beta=1$: standard exponential discounting
    \item $\beta<1$: present bias
    \item The time-inconsistency here comes from comparing future periods; the discounting between today and tomorrow, and between one month from now vs. two months from now, are discounted differently
    \item We hit all future periods with an extra $\beta$
\end{itemize}
\textbf{Example:}

Suppose $\beta=0.9$ and $\delta=1$
\begin{enumerate}
    \item Choose between \$99 in $t=1$ and \$100 in $t=2$
    \item Choose between \$99 in $t=3$ and in \$100 $t=4$
\end{enumerate}
In $t=1$:
\begin{align*}
    U^1&=\delta^{0}u(c_1)=1\times 99=99 \\
    U^1&=\beta\delta^{1}u(c_2)=0.9\times 1\times 100=90 \\
    U^1&=\beta\delta^{2}u(c_3)=0.9\times 1\times 99=89.1 \\
    U^1&=\beta\delta^{3}u(c_4)=0.9\times 1\times 100=90
\end{align*}
In $t=3$:
\begin{align*}
    U^3&=\delta^{0}u(c_3)=\times 1\times 99=99 \\
    U^3&=\beta\delta^{1}u(c_4)=0.9\times 1\times 100=90
\end{align*}

\section{Self-Control}
\begin{itemize}
    \item There are three periods
    \item Income = $Y_1$ (no other income sources in other periods)
    \item There are matching contributions: $M$ times the amount saved by the start of $t=3$
    \item $t=1, 2$, agent must make an allocation decision between savings and consumption
    \item The consumer has quasi-hyperbolic preferences with $\delta=1$ for simplicity and $\beta\in(0,1]$
    \item Assume sophistication: agent knows his future $\beta$ and there is no uncertainty
    \item Utility is given by an instantaneous utility function $u(c_t)$ which is increasing and concave
    \[u'(\cdot)>0 \textrm{ and } u''(\cdot)<0 \]
    \item Agent's maximization problem is as follows:
    \begin{itemize}
        \item In $t=1$:
        \[\max\quad U_1(c_1,c_2,c_3)\equiv u(c_1)+\beta[u(c_2)+u(c_3) ] \]
        \item In $t=2$:
        \[\max\quad U_2(c_2,c_3)\equiv u(c_2)+\beta u(c_3) \]
    \end{itemize}
\end{itemize}
\textbf{No commitment savings}

Solve recursively:
\begin{itemize}
    \item In $t=3$, the agent consumes whatever is left
    \item In $t=2$, solve the following maximization problem:
    \[\max_{c_2}\quad u(c_2)+\beta u\left((Y_1-c_1-c_2)(1+M)\right) \]
    \[u'(c_2)=\beta(1+M)u'\left((Y_1-c_1-c_2)(1+M)\right) \]
    \item In $t=1$, the agent takes the $t=2$ constraint as given and solves:
    \begin{align*}
        \max_{c_1}\quad & u(c_1)+\beta[u(c_2)+u(c_3) ] \\
        \textrm{s.t.}\quad & c_3=(Y_1-c_1-c_2)(1+M) \\
        & u'(c_2)=\beta(1+M)u'(c_3) \\
        & c_1, c_2, c_3 \geq 0
    \end{align*}
    \begin{itemize}
        \item Defining $Y_2\equiv Y_1-c_1$
        \begin{align*}
            u'(c_1)&=\beta\left[u'(c_2)\frac{\textrm{d}c_2}{\textrm{d}Y_2}+
            u'(c_3)\frac{\textrm{d}c_3}{\textrm{d}Y_2} \right] \\
            u'(c_2)&=\beta(1+M)u'(c_3) \\
            c_3&=(Y_1-c_1-c_2)(1+M)
        \end{align*}
        \item Euler equation:
        \[u'(c_1)=\left[\beta\frac{\textrm{d}c_2}{\textrm{d}Y_2}+
        \left(1-\frac{\textrm{d}c_2}{\textrm{d}Y_2}\right) \right]u'(c_2) \]
    \end{itemize}
\end{itemize}

\textbf{Commitment savings}

\begin{itemize}
    \item In $t=1$, the agent would like to set $u'(c_2)=(1+M)u'(c_3)$
    \item When the agent has self-control problems, he is unable to ensure this pattern of consumption, as in $t=2$ he would prefer to set $u'(c_2)=\beta(1+M)u'(c_3)$, which is more than he would like to in $t=1$
    \item The agent solves the problem as a $t=1$ maximization for all periods, which gives the following set of equations for the solution
    \begin{align*}
        u'(c_1)&=\beta u'(c_2) \\
        u'(c_2)&=(1+M) u'(c_3) \\
        c_3&=(Y_2-c_2)(1+M)
    \end{align*}
    \begin{itemize}
        \item If $\beta=1$, commitment savings has no effect
        \item If $\beta=0$, no savings
        \item If $\beta\in(0,1)$, two opposing effects on the impact of commitment on savings
        \begin{itemize}
            \item Without commitment, $t=2$ self will deviate further from optimal consumption in $t=1$. The impact on savings of having a commitment device is larger for increased present bias.
            \item However, $t=1$ self also has a decreasing $\beta$, therefore less of a desire to allocate consumption to later periods.
        \end{itemize}
    \end{itemize}
\end{itemize}

\end{document}
