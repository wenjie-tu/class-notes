\documentclass[a4paper]{article}
\usepackage[utf8]{inputenc}
\usepackage{amsmath,amsfonts,amssymb,amsthm}
%\usepackage{mathtools}
\usepackage{graphicx}
\usepackage{setspace}
% \usepackage[a4paper, total={6in, 8in}]{geometry}
\usepackage[top=2.0cm, left=2.0cm, right=2.0cm, bottom=3.0cm]{geometry}
\usepackage{xcolor}

\renewcommand{\familydefault}{\sfdefault}

\title{%
    Applied Statistical Regression 
}
\author{Wenjie Tu}
\date{Fall Semester 2021}

\setlength{\parindent}{0pt}
\setlength{\parskip}{1em}
%\onehalfspacing
\begin{document}

%\maketitle

\section{Simple Regression}
\begin{itemize}
    \item The LOESS-Smoother is better, more flexible and more robust than the Gaussian Kernel Smoother.
    \item Log-transformation on both response and predictor: this is an estimate for the median of the conditional distribution $y\mid x$, but not the conditional mean $E[y\mid x]$. If we require unbiased fitted values on the original scale, applying a correction factor is required!
    \[\hat{y}=\exp(\hat{y}'+\frac{\hat{\sigma}_E^2}{2}) \]
    \[\hat{y}=\exp(\hat{y}')\cdot\frac{1}{n}\sum_{i=1}^n\exp(r_i') \]
    \item Confidence intervals for the median and prediction intervals by simple transformation:
    \[[lo,up]\to[\exp(lo),\exp(up)] \]
    \begin{itemize}
        \item If one needs a confidence interval for the mean, the theoretical or Duan's smearing correction has to be used
    \end{itemize}
\end{itemize}

\section{Multiple Regression}
\begin{itemize}
    \item \underline{Necessary but not sufficient condition} for the full rank of $X$: $p<n$
    \item Gauss-Markov Theorem: The OLS regression coefficients are unbiased, and they have minimal variance among all estimators that are \underline{linear and unbiased}.
    \item If the errors are i.i.d. and follow a Normal distribution, the estimated regression coefficients and the fitted values will also be normally distributed.
    \item Adjusted R-squared is always (but often irrelevantly) smaller than multiple R-squared.
    \item Individual Parameter Tests
    \begin{itemize}
        \item The p-values of the individual hypothesis tests are based on the assumption that the other predictors remain in the model and do not change. Therefore, you must not drop \underline{more than one} single non-significant predictor at a time.
        \item It can happen that all individual tests do not reject the null hypothesis, although some predictors have a significant effort on the response. \textbf{Reason}: correlated predictors!
        \item Multiple testing problem: when doing many tests, the total type I error (i.e. false rejection) increases and we may observe spuriously significant predictors. (Overall type I error: $1-(1-p)^m$)
    \end{itemize}
    \item Prediction: only interpolation (i.e. prediction within the range of observed y-values works well); extrapolation yields non-reliable results.
    \item Inference with Factor Variable
    \begin{itemize}
        \item In a regression model where factor variables that have \underline{more than 2 levels and/or interaction terms} are present, the \texttt{summary()} function does not provide useful information for variable selection. We have to work with \texttt{drop1()} instead! \texttt{drop1()} performs correct model comparisons and respects the model hierarchy.
    \end{itemize}
    \item Inference with Categorical Predictors
    \begin{itemize}
        \item Do not perform individual hypothesis tests on factors that have \underline{more than 2 levels}, they are meaningless! We use \texttt{anova()}.
    \end{itemize}
    \item Residuals vs. Errors
    \begin{itemize}
        \item The residual random variables $R_i$ share some properties of the errors $E_i$, but not all - there are important differences.
        \item Even in cases where the $E_i$ are uncorrelated and have constant variance, the residuals $R_i$ feature some \underline{estimation-related correlation and non-constant variance}.
        \item The estimation-induced correlation and heteroskedasticity in the residuals $R_i$ is usually very small. Thus, residual analysis using the raw residuals $r_i$ is both useful and sensible.
        \item One can try to improve the raw residual $r_i$ with dividing it by an estimate of its standard deviation
        \[\tilde{r}_i=\frac{r_i}{\hat{\sigma}_E\cdot\sqrt{1-H_{ii}}} \]
        where $H_{ii}$ is the diagonal element of hat matrix
        \begin{itemize}
            \item Standardized Residuals: $\hat{\sigma}_E$ is the residual standard error.
            \item Studentized Residuals: $\hat{\sigma}_E$ is estimated by ignoring the $i$-th data point.
        \end{itemize}
    \end{itemize}
    \item Tukey-Anscombe-Plot: Residuals vs. Fitted
    \begin{itemize}
        \item If $E[E_i]\neq0$, the response/predictor relation might be nonlinear or some important predictors/interaction may be missing
        \item If non-constant variance of $E_i$, the smoother deviates from 0
        \item When is the plot OK?
        \begin{itemize}
            \item the residual scatter around the x-axis without any structure (i.e. constant variance)
            \item the smoother line is horizontal without systematic deviation (i.e. zero expectation)
            \item there are no outliers
        \end{itemize}
        \item Systematic error (i.e. the smoother deviates from the x-axis and hence $E[E_i]\neq0$)
        \begin{itemize}
            \item Log-transformation on the response and/or predictors
            \item Omitted variables (novel variables, higher polynomials, interaction terms)
        \end{itemize}
        \item Non-constant variance
        \begin{itemize}
            \item Transformations!
        \end{itemize}
    \end{itemize}
    \item Normal Plot
    \begin{itemize}
        \item Identifying non-iid or non-Gaussian errors
        \item When is the plot OK?
        \begin{itemize}
            \item No systematic deviation from line which leads to the 1st and 3rd quantile
            \item A few data points that are slightly ``off the line'' near the boundaries are often encountered and usually tolerable
            \item Long-tailed but symmetrical residuals are not optimal either, but often tolerable. Alternative: robust regression!
        \end{itemize}
    \end{itemize}
    \item Scale-Location-Plot: $\sqrt{|\tilde{r}_i|}$ vs. $\hat{y}_i$
    \begin{itemize}
        \item Identifying heteroscedasticity
        \item If $Var(E_i)\neq\sigma_E^2$, use transformations or weighted regression!
        \item When is the plot OK?
        \begin{itemize}
            \item The smoother line runs horizontally along the x-axis, without any systematic deviations.
        \end{itemize}
    \end{itemize}
    \item Leverage-Plot: $\tilde{r}_i$ vs. $H_{ii}$
    \begin{itemize}
        \item Leverage: $H_{ii}$
        \begin{itemize}
            \item Leverage points are different from the bulk of data
            \item The average value for leverage is given by $\dfrac{p+1}{n}$
            \item We say a data point has high leverage if $H_{ii}>\dfrac{2(p+1)}{n}$
        \end{itemize}
        \item Standardized residuals: $\tilde{r}_i$
        \item Cook's Distance
        \[D_i=\dfrac{\sum(\hat{y}_k^{[-i]}-\hat{y}_k)^2}{(p+1)\sigma_E^2}=\dfrac{H_{ii}}{1-H_{ii}}\cdot\dfrac{\tilde{r}_i^2}{p+1} \]
        \begin{itemize}
            \item Cook's Distance can be computed directly \underline{without} fitting the regression $n$ times.
            \item It measures the influence of a data point and depends on leverage and standardized residuals.
            \item Data points with $D_i>0.5$ are called influential. If $D_i>1$, the data point is potentially damaging to the regression problem.
        \end{itemize}
        \item Identifying outliers, leverage points, influential observations and uncritical data points at one and the same time
        \item When is the plot OK?
        \begin{itemize}
            \item No extreme outliers in $y$-direction
            \item High leverage (i.e. $H_{ii}>\dfrac{2(p+1)}{n}$) is always potentially dangerous, especially if it appears in conjunction with large residuals
        \end{itemize}
        \item How to deal with influential observations (i.e. $D_i>0.5$)?
        \begin{itemize}
            \item Check the data for misprints, typos
            \item Influential observations often appear if the input is not suitable (i.e. predictors are extremely skewed, log-transformations were forgotten)
            \item Simply omitting these data points is always a delicate matter and should at least be reported openly
            \item Influential data points tell much about the benefits and limits of a model and create opportunities and ideas for how to improve the model
        \end{itemize}
    \end{itemize}
    \item Partial Residual Plots
    \begin{itemize}
        \item The plot of response $y$ vs. predictor $x_k$ can be deceiving because other predictors also influence the response and thus blur our impression
        \item Partial residual plots show the marginal relation between a predictor $x_k$ and the response $y$, after/when the effect of the other variables is accounted for
        \item When is the plot OK?
        \begin{itemize}
            \item If the red line with the actual fit from the linear model, and the green line of the smoother do not show systematic differences
        \end{itemize}
        \item What to do if not OK?
        \begin{itemize}
            \item Using transformations; including additional predictors; adding interaction terms
            \item Using GAM
        \end{itemize}
    \end{itemize}
    \item Correlated Errors
    \begin{itemize}
        \item If the errors/residuals are correlated
        \begin{itemize}
            \item OLS procedure still leads to \underline{unbiased} estimates of both regression coefficients and fitted value
            \item OLS estimator is \underline{no longer efficient}
            \item The standard errors for the coefficients are biased and wiill inevitably lead to flawed inference results (i.e. hypothesis tests and confidence intervals)
            \item The standard errors can be either too small (majority of cases), or too large
        \end{itemize}
        \item Residuals vs. Time/Index
        \begin{itemize}
            \item When is the plot OK?
            \begin{itemize}
                \item No systematic structure
                \item No long sequences of positive/negative residuals
                \item No back-and-forth between positive/negative residuals
                \item The p-value in Durbin-Watson test is $>0.05$
            \end{itemize}
            \item What to do if not OK?
            \begin{itemize}
                \item Adding omitted variables
                \item Using generalized least squares method (GLS)
                \item Estimated coefficients and fitted values are unbiased but confience intervals and tests are biased!
            \end{itemize}
        \end{itemize}
        \item Autocorrelation of Residuals
        \item Durbin-Watson-Test
        \[DW=\frac{\sum_{i=2}^n(r_i-r_{i-1})^2}{\sum_{i=1}^nr_i^2} \]
        \begin{itemize}
            \item Under the null hypothesis, no correlation. The test-statistic has a $\chi^2$-distribution. The p-value can be computed.
            \item The DW-test is somewhat problematic, because it can only detect simple correlation structure. When more complex dependency exists, it has very low power.
        \end{itemize}
    \end{itemize}
    \item Multicollinearity
    \begin{itemize}
        \item The columns $X$ show strong correlation (but not perfect dependence).
        \begin{itemize}
            \item In these cases, there is a (technically) unique solution, but it is often highly variable and poor for practical use
            \item The estimated coefficients feature large standard errors (wider confidence intervals)
            \item Typical case: the global F-test turns out to be significant, but none of the individual predictors is significant
            \item Extrapolation may yield extremely poor results
        \end{itemize}
        \item Analyzing all pairwise correlation coefficients among predictors will not identify all situations where there is multicollinearity because multicollinearity is not always a pairwise phenomenon
        \item Dealing with multicollinearity
        \begin{itemize}
            \item Amputation: among all colinear predictors, all except one will be discarded
            \item Generating new variables
            \item For factor variables, use generalized VIFs. If the squared $GVIF^{1/(2df)}$ exceed the usual threshold rules (i.e. $>5$, $>10$), then we have overly redundant factor variables among our predictors
        \end{itemize}
    \end{itemize}
    \item Ridge Regression
    \begin{itemize}
        \item Ridge regression requires centered and standardized predictors as the solution is sensitive to location and scale. Furthermore, it is important not to penalize the intercept!
        \item \texttt{lm.ridge()} does \underline{neither provide fitted values nor residuals}. These have to be sorted out by oneself via the design matrix
        \item Residual plots have to be generated by hand too
    \end{itemize}
    \item Variable selection is not a method! The search for the best predictor set is an iterative process. It also involves estimation, inference and model diagnostics
    \item Backward Elimination with p-Values
    \begin{itemize}
        \item The removed variables can still be related to the response. If we run a simple linear regression, they can even be significant. In the multiple linear model, however, there are other better and more informative predictors
        \item In a step-by-step backward elimination, the best model is often missed.
    \end{itemize}
    \item AIC/BIC allow for comparison of models that are \underline{not hierarchical}. But they need to be fitted on exactly the same data points and the response variable needs to be identical
    \item Backward Elimination with AIC/BIC
    \begin{itemize}
        \item The selection stops when AIC/BIC cannot be improved anymore. \underline{Predictors do not need to be significant}
    \end{itemize}
    \item Exhaustive Search
    \begin{itemize}
        \item For finding the model that globally minimizes AIC, a complete search over all $2^p$ models is required. Depending on $n$ and computing speed, this is feasible for $p\approx15-20$
        \item R function \texttt{regsubsets()} of \texttt{library(leaps)} does the job, but \underline{it cannot deal with factor variable correctly}
    \end{itemize}
    \item Guidelines for Variable Selection
    \begin{itemize}
        \item Factor variables either remain with all their dummies, or are entirely removed. There is no in-between solution
        \item If interaction terms are present, then the main effects for these variables must be in the models as well
        \item In case of polynomial terms, all lower order terms must be used as well
        \item Stick to these rules when using manual selection procedures
        \item \texttt{regsubsets()} does not obey to these rules, while \texttt{step(), drop1(), add1()} do
    \end{itemize}
    \item LASSO
    \begin{itemize}
        \item Coefficients are artificially shrunken towards zero and hence \underline{biased}. The benefit is that they are less variable (smaller standard errors)
        \item No explicit closed-form solution to LASSO.
        \item In contrast to the OLS and Ridge estimators, the solution is found via numerical optimization. Using the Coordinate Descent procedure in R allows for finding the solution up to problem size around $np\approx10^6$
        \item In contrast to the OLS and Ridge estimators, LASSO is not a linear estimator. There is \underline{no hat matrix} $H$ such that $\hat{y}=Hy$
        \item As a result, \underline{the concept of degrees of freedom does not exist for LASSO} and there is no trivial procedure for choosing the optimal penalty parameter $\lambda$ 
        \item Inference on the fitted model is difficult, or even impossible (biased). One can compare standardized coefficients
        \item The standard LASSO \underline{only works} with numerical predictors. Extension to factor variables exist (see Group LASSO)
    \end{itemize}
    \item Cross Validation
    \begin{itemize}
        \item The only key point is that the \underline{same response variable} is predicted.
        \item We can perform cross validation on datasets with \underline{different number of observations, or even on different datasets}
        \item Cross validation allows for comparison of models that are \underline{not hierarchical}, and that can be arbitrarily different
        \item It is possible to infer the effect of response variable transformations, LASSO, Ridge, robust procedures
        \item AIC/BIC and Adjusted R-squared \underline{do not work} if
        \begin{itemize}
            \item We want to investigate whether we obtain better prediction from a model with transformed response or not
            \item We want to check whether excluding data points from the fit yields better predictions from the entire sample
            \item We want to compare performance of alternative methods such as LASSO, Ridge or Robust Regression
        \end{itemize}
    \end{itemize}
    \item Significance vs. Relevance
    \begin{itemize}
        \item The larger a sample, the smaller the p-values for the very same predictor effect. Do not confuse small p-values with an important predictor effect!
        \item With large datasets,
        \begin{itemize}
            \item statistically significant results which are practically useless
            \item most predictors have influence, thus $\beta_j=0$ hardly ever holds
            \item the point null hypothesis is thus usually wrong in practice
        \end{itemize}
        \item Absence of Evidence $\neq$ Evidence of Absence
        \item Measuring the relevance of predictors
        \begin{itemize}
            \item Maximum effect of a predictor variable on the response:
            \[\lvert\beta_j\cdot(\max_ix_{ij}-\min_ix_{ij}) \rvert \]
            \item Standardizing all predictors to mean zero and unit variance, which makes the coefficients $\beta_j$ directly comparable
            \item More sophisticated approach - LMG criterion
        \end{itemize}
    \end{itemize}
\end{itemize}

\section{Extending the Linear Model}
\begin{itemize}
    \item Penalized Regression Splines
    \begin{itemize}
        \item Fitting Penalized Regression Splines is a parametric problem that can be solved analytically in closed form via Generalized Ridge Regression
        \item Choosing $\lambda$ such that the Generalized Cross Validation Score (GCV) is minimized
    \end{itemize}
    \item \texttt{smooth.spline()} vs. \texttt{gam()}
    \begin{itemize}
        \item \texttt{smooth.spline()}
        \begin{itemize}
            \item Uses all knots (if \texttt{all.knots=TRUE})
            \item Spends more degrees of freedom for the fit
            \item Only works with one single predictor variable
        \end{itemize}
        \item \texttt{mgcv::gam()}
        \begin{itemize}
            \item By default uses thin plate regression splines
            \item Method can be adjusted by \texttt{s(xx, "cr")}, not recommended
            \item By default does not use all knots (i.e. uses a basis that has a limited dimension for saving computing time for obtaining good performance in multiple predictor settings)
        \end{itemize}
    \end{itemize}
\end{itemize}

\section{Exam S21}
\begin{itemize}
    \item Confidence Intervals vs. Prediction Intervals
    \begin{itemize}
        \item The CI characterizes the variability in the fitted value, not the future response value
        \item The PI says that if we use another sample from the same population, the response values will be lying within this interval
        \item The PI accounts for the scatter of the data points around the regression line, thus PI is always wider than the corresponding CI
    \end{itemize}
    \item \texttt{gam()}: which variable is penalized the most?
    \begin{itemize}
        \item The variable with smaller \texttt{edf} is penalized harder
    \end{itemize}
    \item In a regression model where factor variables that have $>2$ levels and/or interaction terms are present, the \texttt{summary} function does not provide useful information for variable selection. Alternatives: \texttt{drop1()}, \texttt{anova()}
    \item Because the first model is fully parametrized with all pairwise interactions terms, hence can be separated into models for each species.
    \item It is possible that the model obtained with \texttt{direction='both'} is larger than the one obtained by using \texttt{direction='forward'}
    \item Under Gaussian errors the OLS estimator is the maximum likelihood estimator.
    \item Split the data into $K$ roughly equal-sized parts in cross validation (True).
    \item For any $\lambda\in(0,\infty)$, the minimizer of the smoothing splines is always given by cubic splines with knots at \underline{each data point}.
    \item Logistic regression model can be viewed as a model where we try to find a relation between the conditional expected value of response y and the predictors.
\end{itemize}

\section{Exam S20}
\begin{itemize}
    \item $exp$ makes the distribution right-skewed and therefore the mean is bigger than the median
    \item By comparing this formula to the one for the AIC (above), one can notice that for any $n>7$, $\log(n)>2$, hence BIC penalizes for model size stronger than AIC
    \item If the observed response of a data point is increased by 1, then the new fitted value of this data point is equal to the old fitted value times the leverage of the data point
    \item The \texttt{anova} function can be employed to test two Gaussian Kernel smoother fits with different bandwidths.
    \item When fitting a model using regression splines, setting a knot at each data point would result in overfitting (True)
    \item Generally, for a twice differentiable piecewise cubic fit with $k$ knots, there are $k+4$ basis functions.
\end{itemize}

\section{Exam W20}
\begin{itemize}
    \item A global test of the fitted model against the null model in logistic regression - $\chi_{p-q}^2$
    \item If we incorporate the dispersion parameter in binomial model, the following features of fitted object do not change:
    \begin{itemize}
        \item the estimated coefficients
        \item the fitted values
        \item any prediction
        \item degrees of freedom
        \item residual deviance
        \item AIC score of the model
        \item log-oods
        \item $\hat{y}(=\hat{p}n)$
    \end{itemize}
    \item The running mean smoother is not robust to outliers
    \item The function \texttt{step()} can not perform variable selection based on $p$-values.
    \item We do not need values of the response variable to compute the leverages of the individual data points
    \item The Cook’s Distance tells us how strongly a data point may force the regression line to run through it (False)
    \item It is mandatory to use a \underline{non-robust smoother} in the TA plot in logistic regression
    \item Type I and Type II error
\end{itemize}

\section{Exam S19}
\begin{itemize}
    \item Even in a well specified least squares regression we expect the residuals to be correlated.
    \item Diagnostic plots
\end{itemize}

\section{Exam W19}
\begin{itemize}
    \item VIFs are strictly more informative than pairwise correlation plots; If predictors are correlated pairwise, their VIF is high, but the reverse implication does not hold.
    \item Ridge regression is not appropriate as it does not force predictors out from the model and hence \underline{does not change}  the VIFs.
    \item Leverage of point $i$ is given by $H_{ii}$, where $H$ is the hat matrix.
    \item Sum of the leverages of all data points:
    \[\sum_iH_{ii}=Tr(H)=Tr(X(X^TX)^{-1}X^T)=Tr((X^TX)^{-1}X^TX)=Tr(I)=p \]
    \item Interaction term between categorical variable and numerical variable
\end{itemize}

\end{document}